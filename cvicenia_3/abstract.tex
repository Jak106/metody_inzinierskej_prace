% Metódy inžinierskej práce

\documentclass[10pt,oneside,english,a4paper]{article}

\usepackage[slovak]{babel}
\usepackage[IL2]{fontenc}
\usepackage[utf8]{inputenc}
\usepackage{graphicx}
\usepackage{url} % príkaz \url na formátovanie URL
\usepackage{hyperref} % odkazy v texte budú aktívne (pri niektorých triedach dokumentov spôsobuje posun textu)

\usepackage{cite}
%\usepackage{times}

\pagestyle{headings}

\title{Protection of private data against web scraping} % meno a priezvisko vyučujúceho na cvičeniach

\author{Jakub Ján Zuber\\[2pt]
	{\small Slovak technological university in Bratislava}\\
	{\small Faculty of informatics and informatical technologies}\\
	{\small \texttt{xzuber@stuba.sk}}
	}

\date{\small 1.10.2023} % upravte

\begin{document}

\maketitle

\begin{abstract}

My work is going to be mainly focused on protection of important people within companies against this technique. People employed at positions of CEO, CFO, COO are often targets of whaling by hacking/scamming groups. This is caused by their huge influence and access to bank acounts of their big companies. Due to the number of interviews and public information available it is hard for them to keep their lives private. After introducing the problems, I will introduce couple of already existing solutions and try to design algorithms capable of detecting webscraping tools/bots. Together with that, I will evaluete some of the training methods that are used to teach privacy protection to this group of people. In my work, I plan on including interviews with people in leading positions as well. Their answers will be evaluated in conjunction with training methods.

\end{abstract}

% týmto sa generuje zoznam literatúry z obsahu súboru literatura.bib podľa toho, na čo sa v článku odkazujete

\bibliography{literatura}
\bibliographystyle{apalike}

\end{document}
